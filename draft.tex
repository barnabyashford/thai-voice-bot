\documentclass{article}
\usepackage{graphicx} % Required for inserting images

\title{Leveraging LLM in developing “Virtual Friend” for Idle Conversation}
\author{Worrapat Cheng \\ \textit{\small Language and Information Technology Major,} \\ \textit{\small Faculty of Arts,  Chulalongkorn University}}

\date{}

\begin{document}

\maketitle

\begin{center}
\begin{abstract}
    Please, brain get something here.
\end{abstract}
\end{center}

\small \textbf{Keywords: } Artificial Intelligence, Automatic Speech Recognition, Chatbot, Large Language Model, Multimodal, Natural Language Processing, Text-to-Speech

\normalsize

\section{Introduction}

Since the advent of Large Language Models (LLM), many efforts have been made to either improve LLM's performance at language generation and comprehension, or devise techniques to optimise resource usage of LLM. As LLM evolves over the years, there have also been efforts to extend the linguistic capability beyond the boundary of text. This project aims to do exactly that, by incorporating different models of different tasks along with LLM to form a system that differs from LLM's typical task of Textual Chat Bot, a "Voice Bot".

\section{Problem Definition}

LLMs are trained on textual data hence their proficiency in generating and understanding natural language in text form. 

\section{Architecture}

\section{Implementation}

\section{Testing \& Evaluation}

\section{Discussion}

\section{Conclusion}

\end{document}
